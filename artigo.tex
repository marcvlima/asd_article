\documentclass[
	% -- opções da classe memoir --
	article,			% indica que é um artigo acadêmico
	11pt,				% tamanho da fonte
	oneside,			% para impressão apenas no verso. Oposto a twoside
	a4paper,			% tamanho do papel. 
	% -- opções da classe abntex2 --
	%chapter=TITLE,		% títulos de capítulos convertidos em letras maiúsculas
	%section=TITLE,		% títulos de seções convertidos em letras maiúsculas
	%subsection=TITLE,	% títulos de subseções convertidos em letras maiúsculas
	%subsubsection=TITLE % títulos de subsubseções convertidos em letras maiúsculas
	% -- opções do pacote babel --
	english,			% idioma adicional para hifenização
	brazil,				% o último idioma é o principal do documento
	sumario=tradicional
]{abntex2}

% ---
% Pacotes fundamentais 
% ---
\usepackage{lmodern}			% Usa a fonte Latin Modern
\usepackage[T1]{fontenc}		% Selecao de codigos de fonte.
\usepackage[utf8]{inputenc}		% Codificacao do documento (conversão automática dos acentos)
\usepackage{indentfirst}		% Indenta o primeiro parágrafo de cada seção.
\usepackage{nomencl} 			% Lista de simbolos
\usepackage{color}				% Controle das cores
\usepackage{graphicx}			% Inclusão de gráficos
\usepackage{microtype} 			% Para melhorias de justificação

% ---
% Pacotes glossaries
% ---
\usepackage[subentrycounter,seeautonumberlist]{glossaries}
% ---

% ---
% Pacotes de citações
% ---
\usepackage[brazilian,hyperpageref]{backref}	 % Paginas com as citações na bibl
\usepackage[alf]{abntex2cite}	% Citações padrão ABNT
% ---


\titulo{Camada Lógica de Cache na Arquitetura Aplicações Móveis Multi Plataforma Baseada em Tecnologias Abertas}
\autor{Marcus Vinícius de Lima\thanks{\url{marcvlima@gmail.com}}}
\local{Brasil}

% ---
% Configurações de aparência do PDF final

% alterando o aspecto da cor azul
\definecolor{blue}{RGB}{41,5,195}

% informações do PDF
\makeatletter
\hypersetup{
	%pagebackref=true,
	pdftitle={\@title}, 
	pdfauthor={\@author},
	pdfsubject={Modelo de artigo científico com abnTeX2},
	pdfcreator={LaTeX with abnTeX2},
	pdfkeywords={abnt}{latex}{abntex}{abntex2}{atigo científico}, 
	colorlinks=true,       		% false: boxed links; true: colored links
	linkcolor=blue,          	% color of internal links
	citecolor=blue,        		% color of links to bibliography
	filecolor=magenta,      		% color of file links
	urlcolor=blue,
	bookmarksdepth=4
}
\makeatother
% --- 

% ---
% compila o indice
% ---
\makeindex
% ---

% ---
% entradas do glossario
% ---
 \newglossaryentry{sdk}{
 	name={SDK},
 	plural={SDKs},
 	description={Software Development Kit - Kit de Desenvolvimento de Software. Conjunto de ferramentas empregadas no processo de desenvolvimento de software}
}
\newglossaryentry{back-end}{
	name={Back-end},
	plural={Back-ends},
	description={Num sistema que emprega a arquitetura cliente servidor, o termo back end e empregado para referenciar o conjunto de camadas que constituem a parte servidor}
}

\makeglossaries
% ---
% Altera as margens padrões
% ---
\setlrmarginsandblock{3cm}{3cm}{*}
\setulmarginsandblock{3cm}{3cm}{*}
\checkandfixthelayout
% ---

% --- 
% Espaçamentos entre linhas e parágrafos 
% --- 

% O tamanho do parágrafo é dado por:
\setlength{\parindent}{1.3cm}

% Controle do espaçamento entre um parágrafo e outro:
\setlength{\parskip}{0.2cm}  % tente também \onelineskip

% Espaçamento simples
\SingleSpacing

% ---
% Exemplo de configurações do glossairo
\renewcommand*{\glsseeformat}[3][\seename]{\textit{#1}  
	\glsseelist{#2}}
% ---

\begin{document}
\selectlanguage{brazil}
\maketitle

\selectlanguage{english}
\begin{abstract}
Abstract text.
\end{abstract}

\selectlanguage{brazil}

\begin{abstract}
Resumo em português.
\end{abstract}

\tableofcontents

\section{Introdução} \label{introduction}
\addcontentsline{toc}{section}{SecIntrodução}

\section{Condutores Arquiteturais para uma Aplicação Móvel}
\addcontentsline{toc}{section}{SecCondutoresAplicacaoMovel}
Um dos grandes fatores para garantia de qualidade no desenvolvimento de aplicações consiste na correta identificação de seus condutores arquiteturais e os mecanismos que irão assegurar uma implementação que os atenda \cite{bachmann2001introduction}.

Condutores arquiteturais são o uma combinação de requisitos funcionais, requisitos de qualidade e restrições que influenciam na determinação das características arquiteturais da aplicação \cite{bachmann2001introduction}.

No contexto de aplicações móveis os seguintes requisitos arquiteturais são elencados:
\begin{enumerate}
	\item Suporte para conexões instáveis \cite{rathore2007overview}:
	
	As tecnologias de telecomunicações móveis implementadas até o presente momento não asseguram instabilidade e garantem a manutenção de conetividade dos dispositivos móveis. Dessa maneira a aplicação deve estar preparada para eventuais falhas de conectividade.
	
	\item Suporte para baixa largura de conexão \cite{rathore2007overview}:
	
	Mesmo com o avanço tecnológico as redes de telefonia móveis em sua grande maioria oferecem velocidades de conexão inferiores as plataformas fixas. Portanto aplicações uma aplicação móvel deve estar preparada para assegurar que as operações sejam realizadas mesmo em cenários de baixa largura de banda, como por exemplo otimizando a quantidade de dados trafegados entre cliente servidor.
	
	\item Suporte para velocidades de \emph{upstream} reduzidas \cite{rathore2007overview}:
	
	Em redes móveis a velocidade de conexão na direção cliente/servidor é inferior a velocidade de conexão na direção servidor/cliente. Deste as requisições de uma aplicação móvel para o servidor devem ser otimizadas para conter a menor quantidade de dados possível.
	
	\item Otimização de bateria \cite{rathore2007overview}:
	
	A fonte de energia em dispositivos móveis é um dos recursos que deve ser prioritariamente preservado, pois está diretamente relacionado ao tempo de autonomia do dispositivo. Uma aplicação móvel portanto deve ser projetada de modo a consumir a menor quantidade de energia possível, como por exemplo requerendo uma menor quantidade de memória RAM para sua execução.
	
	\item Espaço de armazenamento reduzido \cite{tiffany2008guide}:
	
	O espaço de armazenamento não volátil em dispositivos móveis é consideravelmente menor do que em dispositivos fixos. Portanto, a aplicação deve otimizar a quantidade de dados que é armazenada para o seu funcionamento.
	
\end{enumerate}

Como este trabalho tem por objetivo a elaboração de uma arquitetura genérica apenas os condutores arquiteturais mais importantes de uma aplicação móvel estão sendo considerados. Ao considerar uma aplicação específica, os requisitos arquiteturais devem ser definidos de maneira mais detalhada. Uma das metodologias que poderia ser empregada para definição de requisitos arquiteturais, está definida em \cite{eeles2005capturing}.

\section{Aplicações Móveis Multiplataforma}
\addcontentsline{toc}{section}{SecAplicacoesMultiplataforma}
Recentemente as aplicações móveis tem apresentado um crescimento expressivo. Desde a queda de popularidade das plataformas mobile BlackBerry, Bada e Symbian, iOS e Android tem adquirido uma força expressiva no mercado \cite{dalmasso2013survey}.

Porém, a diversidade das plataformas e a variedade de \gls{sdk} e ferramentas disponíveis criou uma segmentação desafiada para o mercado de desenvolvimento de aplicações móveis No processo de desenvolvimento uma série de fatores precisam ser considerados como por exemplo a experiência de usuário oferecida, estabilidade do framework, facilidade de manutenção, o custo de desenvolvimento multiplataforma e o tempo de mercado da aplicação \cite{dalmasso2013survey}.

A grande maioria dos desenvolvedores e organizações querem atingir a maior fatia de mercado com seus aplicativos. Para tal é necessário que o aplicativo esteja disponível nas principais plataformas mobile como o iOS, Android e Windows Phone. 

Porém o desenvolvimento para estas plataformas requerem amplo conhecimento do desenvolvedor sobre as mesmas e sobre os seus respectivos \gls{sdk}, o que eleva o custo de desenvolvimento e a complexidade do processo. Para contrapor a esta barreira imposta pela pluralidade de plataformas de desenvolvimento móvel entra em cena a abordagem multiplataforma, capaz de reduzir o esforço de desenvolvimento e o tempo para lançamento de aplicativos no mercado \cite{dalmasso2013survey}.

Como citado em \cite{shehab2014reducing} a vantagem de custo e tempo de mercado proporcionado por esta abordagem está promovendo o surgimento de distintas plataformas que a adotam. Essas plataformas estão competindo pela adoção de desenvolvedores em características como o número de plataformas nativas suportadas, acesso a funções nativas dos dispositivos móveis, facilidade de acesso ao \gls{back-end} e por melhorias de performance. 

\section{Cache de Aplicações Móveis}
\addcontentsline{toc}{section}{SecCacheMovel}

\section{Princípios de Separação de Responsabilidades da Aplicação em Camadas Lógicas}
\addcontentsline{toc}{section}{SecCacheMovel}

\section{Arquitetura Proposta}
\addcontentsline{toc}{section}{SecArquiteturaProposta}

\subsection{Importancia de Tecnologias Abertas}
\addcontentsline{toc}{subsection}{SubSecTecnologiasAbertas}

\subsection{Apache Cordova}
\addcontentsline{toc}{subsection}{SubSecApacheCordova}

\subsection{SQLite}
\addcontentsline{toc}{subsection}{SubSecSQLite}

\subsection{JavaScript}

\subsection{AngularJS}
\addcontentsline{toc}{subsection}{SubSecAngularJS}

\subsection{Prova de Conceito da Arquitetura Proposta}
\addcontentsline{toc}{subsection}{SubSecPOC}

\section{Conclusão} \label{introduction}
\addcontentsline{toc}{section}{SecConclusao}

\section{Trabalhos Futuros} \label{introduction}
\addcontentsline{toc}{section}{SecTrabalhosFuturos}

\bibliography{artigo}{}

% ----------------------------------------------------------
% Glossário
% ----------------------------------------------------------

% ---
% Define nome e preâmbulo do glossário
% ---
\renewcommand{\glossaryname}{Glossário}
%\renewcommand{\glossarypreamble}{Esta é a descrição do glossário.\\ \\}

% ---
% Traduções para o ambiente glossaries
% ---
\providetranslation{Glossary}{Glossário}
\providetranslation{Acronyms}{Siglas}
\providetranslation{Notation (glossaries)}{Notação}
\providetranslation{Description (glossaries)}{Descrição}
\providetranslation{Symbol (glossaries)}{Símbolo}
\providetranslation{Page List (glossaries)}{Lista de Páginas}
\providetranslation{Symbols (glossaries)}{Símbolos}
\providetranslation{Numbers (glossaries)}{Números} 
% ---

% ---
% Imprime o glossário
% ---
\cleardoublepage
\phantomsection
\addcontentsline{toc}{chapter}{\glossaryname}
% \glossarystyle{index}
% \glossarystyle{altlisthypergroup}
\glossarystyle{tree}
\printglossaries
% ---
% ---

\end{document}